\documentclass{article}

\usepackage{hyperref}
\hypersetup{
    colorlinks=true,
    urlcolor=magenta
}

\title{Annual Progress Report, Year 4}
\author{Kate Storey-Fisher \\
Thesis Committee: David W. Hogg, Jeremy Tinker, \\
Michael Blanton, David Grier \\
Slides available at \href{https://docs.google.com/presentation/d/1hqaPXNtvDq3Bnf1CegbAA840C_V2PaNjywn1dg2yrPM/edit?usp=sharing}{this URL}
}

\begin{document}
\maketitle


\section{Research Progress}

\subsection{Two-point statistics without bins}

\hspace{\parindent} \textbf{Status:}
We completed this project and submitted the paper presenting the method and implementation; it was published in February.
This project is in collaboration with David W. Hogg.

\textbf{Summary:}
We developed the Continuous-Function Estimator (CFE), a continuous estimator for the two-point correlation function (2pcf), the most important statistic for cosmological inference.
The CFE projects the data, the positions of large-scale structure tracers, onto a set of continuous basis functions.
It computes the linear least-squares estimate of the 2pcf in the limit of a large random catalog, in terms of the cell occupation number.
The CFE is a generalization of the Landy-Szalay estimator; for the choice of tophat basis functions, it reduces to Landy-Szalay.
It obviates bias/variance issues with binning, requires fewer mock catalogs for covariance matrix estimation, and allows for direct estimation of the parameters of interest.

We demonstrate the CFE with an application to inferring cosmological distance scales from the baryon acoustic oscillation (BAO) feature.
We use a set of five basis functions adapted from the standard BAO fitting function, linearized around the parameter of interest, the scale dilation parameter.
The CFE finds the best-fit linear combination of these basis functions, and directly outputs an estimate of the scale dilation parameter.
We show that we accurately and precisely recover the value of this parameter, using a suite of lognormal mock catalogs.

\textbf{Future:}
We are currently working with an NYU physics undergraduate, Abby Williams, on an application of the CFE.
We are investigating long-wavelength gradients in the large-scale structure.
This project is motivated by observations of a hemispherical power asymmetry in the CMB, as well as a claim of anisotropy in the Type Ia supernovae distribution.
Our goal is to use the CFE's ability to incorporate other properties of the tracer into its measurement of clustering; we will include a dependence on spatial position in addition to separation.
We have generated mock catalogs with injected gradients, and have applied the CFE to recover these, as well as a standard least-squares method.
Williams will present this work at the 238th meeting of the American Astronomical Society this June, and we will develop it into a journal publication.

We have many other ideas of applications of the CFE that we hope to explore.
These include reformulating standard binned analyses, such as anisotropic BAO analyses and the 2D correlation function to measure redshift space distortions, and using a Fourier basis to construct an estimator closely related to the power spectrum.
We can also perform direct estimations of other cosmological quantities, such as the growth rate of structure or local primordial non-Gaussianity, or even fitting a full cosmological model directly with the CFE.
We plan to collaborate with researchers performing state-of-the-art cosmological analyses and/or working with the current large surveys. 

Finally, we hope to apply the CFE to make projections for upcoming NASA surveys (Euclid or WFIRST), in line with my NASA grant.
This will require obtaining or generating the necessary mock catalogs, and perhaps collaborating with experts on cosmological analyses for these surveys.

\subsection{Emulation of clustering statistics}

\hspace{\parindent} \textbf{Status:}
We are actively working on this research and have made significant progress.
The main aspects of the project are completed but we are working through some difficulties, described below.
This work is advised by Jeremy Tinker and in collaboration with the Aemulus project.

\textbf{Summary:}
The goal of this project is to extract more cosmological information from small-scale clustering by combining emulators of various clustering statistics.
I have constructed Gaussian Process emulators of the monopole, the projected correlation function, the underdensity probability function, and the marked correlation function; we are interested in whether the latter two add useful information about the parameters of interest.
In particular, we are investigating halo occupation distribution and assembly bias parameters in addition to cosmological parameters, and we expect these void and density statistics to be useful for these constraints.

I constructed a covariance matrix for recovery tests using the Aemulus suite of simulators, and performed MCMC to obtain parameter constraints.
The constraints are generally very good, and we are seeing that the void and density statistics add significant information to standard statistics.
However, we are finding some spurious modes in the recovered posteriors.
We have investigated whether these were due to a variety of issues, including poor MCMC chains, bad emulators, and a noisy covariance matrix, but have not found a clear culprit.

\textbf{Future:}
We will use the helpful feedback from my thesis committee to continue working on this issue.
After we address this issue, we will write up a publication on our results.
We next plan to apply this method to the latest survey data and obtain parameter constraints.

\subsection{Anomaly detection in galaxy images}

\subsection{Future projects}

I am beginning to work on and think about future directions for my research; here I briefly present some of these potential new projects.

\textbf{Cosmology with Graph Neural Networks}

\section{Publications}

\textbf{Published:}
\begin{itemize}
\item Storey-Fisher, K. and Hogg, D.W., 2021. Two-point statistics without bins: A continuous-function generalization of the correlation function estimator for large-scale structure. The Astrophysical Journal, 909(2), p.220. arXiv:2011.01836
\item Storey-Fisher, K., Huertas-Company, M., Ramachandra, N., Lanusse, F., Leauthaud, A., Luo, Y. and Huang, S., 2020. Anomaly Detection in Astronomical Images with Generative Adversarial Networks. NeurIPS 2020 Machine Learning and the Physical Science Workshop, arXiv:2012.08082.
\item Margalef-Bentabol, B., Huertas-Company, M., Charnock, T., Margalef-Bentabol, C., Bernardi, M., Dubois, Y., Storey-Fisher, K. and Zanisi, L., 2020. Detecting outliers in astronomical images with deep generative networks. Monthly Notices of the Royal Astronomical Society, 496(2), pp.2346-2361. arXiv:2003.08263
\end{itemize}

\noindent \textbf{Submitted / In Preparation:}
\begin{itemize}
\item Storey-Fisher, K., Huertas-Company, M., Ramachandra, N., Lanusse, F., Leauthaud, A., Luo, Y., Huang, S., and Prochaska, J.X. Anomaly Detection in Hyper Suprime-Cam Galaxy Images with Generative Adversarial Networks (Submitted to MNRAS)
\item Appleby, S., Davé, R., Sorini, D., Storey-Fisher, K. and Smith, B., 2021. The Low Redshift Circumgalactic Medium in Simba. arXiv:2102.10126. (Submitted to MNRAS)
\item Zhai, Z., Tinker, J., Banerjee, A., Becker, M.R., DeRose, J., Guo, H., Mao, Y.Y., McClintock, T., McLaughlin, S.,  Storey-Fisher, K., Rozo, E., Wechsler, R. The Aemulus Project VI: Cosmological constraint from small scale clustering of BOSS galaxies (In prep.)
\item Storey-Fisher, K., Tinker, J., Zhai, Z., Banerjee, A., Becker, M.R., DeRose, J., Guo, H., Mao, Y.Y., McClintock, T., McLaughlin, S., Rozo, E., Wechsler, R. The Aemulus Project VII: Emulation of Void Statistics and the Marked Correlation Function (In prep.)
\end{itemize}

\section{Grants}

\begin{itemize}
\item James Arthur Graduate Award Fellowship, June--August 2020, \$8,750
\item NASA Future Investigators in Earth and Space Science Technology (FINESST), September 2020--August 2023, \$135,000
\end{itemize}

\section{Seminars}

\end{document}